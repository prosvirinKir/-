\thispagestyle{empty}

\begin{small}%
Ввиду значительного объема грузоперевозок железнодорожным транспортом в Российской Федерации поддержание железнодорожной инфраструктуры в~исправном состоянии является критически важной хозяйственной задачей.
Цель настоящей выпускной квалификационной работы~--- построение предиктивной модели на~основе нейронных сетей, призванной прогнозировать опасный отказ верхних строений железнодорожного пути.
Исследована применимость различных архитектурных решений: полносвязной нейронной сети, рекуррентных нейронных сетей на~основе сети Элмана (RNN) и~с~управляемыми рекуррентными блоками (GRU), автокодировщика.
Полученные результаты превосходят по~качеству предсказаний существующие модели на~основе градиентного бустинга.
Достигнут практический порог по~точности предсказаний модели с~учетом зашумленности входных данных.\par
\end{small}
