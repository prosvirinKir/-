\chapter*{Введение}
\addcontentsline{toc}{chapter}{Введение}

\vspace{-3mm}
Объем грузоперевозок в~Российской Федерации железнодорожным транспортом исчисляется 6,38~млрд~тонн или 755~млрд~руб. в~год (2020~г., \cite{seanews}).
Интенсивный грузооборот по~железным дорогам, а~также неблагоприятные климатические условия на~большей части
территории страны вызывают быстрое исчерпание ресурса верхних строений железнодорожного пути.
\emph{Поддержание инфраструктуры в~исправном состоянии~--- это~критически важная хозяйственная задача.}

Для сбора сведений о степени износа железнодорожных путей используют специальные поезда.
Однако проведение ремонта всех участков в отличном от идеального состоянии не только экономически нецелесообразно, но и принципиально невозможно.
Необходимо выделять потенциально наиболее опасные участки для своевременного проведения ремонтных мероприятий.

Нейронные сети в настоящее время широко распространены в решении различных классов задач, таких как классификация образов, кластеризация объектов, аппроксимация функций, предсказания свойств и событий.

Цель настоящей выпускной квалификационной работы~--- \emph{построить предиктивную модель на основе нейронных сетей,
призванную прогнозировать опасный отказ верхних строений железнодорожного пути.} По входным данным за
определенный промежуток времени модель должна предсказывать вероятность отказа железнодорожного
пути в следующем месяце.

Для достижения указанной цели необходимо решить следующие учебно-исследовательские задачи:
% Очень неконкретно
\begin{itemize}[itemsep=.5ex, wide]
	\item рассмотреть существующие модели, которые в данный момент используются для решения данной проблемы;
	\item придумать архитектуру предиктивной модели на основе нейронных сетей;
	\item проверить, смогут ли модели на основе нейронных сетей решать данную проблему лучше, чем существующие аналоги;
	\item провести серии экспериментов, на основе которых можно будет судить об итоговом качестве моделей;
	\item проанализировать полученные результаты и сделать\linebreak вывод о целесообразности использования данных моделей.
\end{itemize}

Внимание автора к данной теме обусловлено желанием применить известные методы машинного обучения в решении актуальной прикладной (хозяйственной) проблемы.

В первой главе настоящей работы описаны используемые архитектуры нейронных сетей,
разъяснены основные принципы работы каждой из моделей, рассказано об их преимуществах и недостатках.
Во второй главе приведены архитектуры сетей, предложенные автором данной работы, 
рассказано о~ключевых этапах подготовки данных, проанализированы полученные результаты.
В~заключении эти результаты обобщены, обозначены возможные направления дальнейших исследований.
